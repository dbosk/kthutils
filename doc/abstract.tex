This package provides various utilities for automation at KTH.
It provides the following modules:
\begin{description}
  \item[kthutils.ug] Access the UG editor through Python.
  \item[kthutils.participants] Read expected course participants through 
    Python.
\end{description}

We also provide a command-line interface for the modules.
This means that the functionality can be accessed through both Python and the 
shell.

\paragraph{An example}

We want to add the user \texttt{dbosk} as teacher in the 
group
\begin{center}
  \texttt{edu.courses.DD.DD1317.20232.1.teachers}.
\end{center}

In Python, we would do
\begin{minted}{python}
import kthutils.credentials
import kthutils.ug

ug = kthutils.ug.UGsession(*kthutils.credentials.get_credentials())

group = ug.find_group_by_name("edu.courses.DD.DD1317.20232.1.teachers")
user = ug.find_user_by_username("dbosk")

ug.add_group_members([user["kthid"]], group["kthid"])
\end{minted}

In the shell, we would do
\begin{minted}{bash}
kthutils ug members add edu.courses.DD.DD1317.20232.1.teachers dbosk
\end{minted}

\paragraph{Installation and documentation}

Install the tools using \texttt{pip}:
\begin{minted}{bash}
python3 -m pip install -U kthutils
\end{minted}

You can read the documentation by running \texttt{pydoc} on the package:
\begin{minted}{bash}
python3 -m pydoc kthutils
\end{minted}
